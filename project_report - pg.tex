\documentclass{report}
\usepackage{setspace}
%\usepackage{subfigure}

\pagestyle{plain}
\usepackage{amssymb,graphicx,color}
\usepackage{amsfonts}
\usepackage{latexsym}
\usepackage{a4wide}
\usepackage{amsmath}

\newtheorem{theorem}{THEOREM}
\newtheorem{lemma}[theorem]{LEMMA}
\newtheorem{corollary}[theorem]{COROLLARY}
\newtheorem{proposition}[theorem]{PROPOSITION}
\newtheorem{remark}[theorem]{REMARK}
\newtheorem{definition}[theorem]{DEFINITION}
\newtheorem{fact}[theorem]{FACT}

\newtheorem{problem}[theorem]{PROBLEM}
\newtheorem{exercise}[theorem]{EXERCISE}
\def \set#1{\{#1\} }

\newenvironment{proof}{
PROOF:
\begin{quotation}}{
$\Box$ \end{quotation}}



\newcommand{\nats}{\mbox{\( \mathbb N \)}}
\newcommand{\rat}{\mbox{\(\mathbb Q\)}}
\newcommand{\rats}{\mbox{\(\mathbb Q\)}}
\newcommand{\reals}{\mbox{\(\mathbb R\)}}
\newcommand{\ints}{\mbox{\(\mathbb Z\)}}

\linespread{1.5}

%%%%%%%%%%%%%%%%%%%%%%%%%%


\title{  	{ \includegraphics[scale=.5]{ucl_logo.png}}\\
{{\Huge Project Title}}\\
{\large Optional Subtitle}\\
		}
\date{Submission date: Day Month Year}
\author{Your name\thanks{
{\bf Disclaimer:}
This report is submitted as part requirement for the MY DEGREE at UCL. It is
substantially the result of my own work except where explicitly indicated in the text.
\emph{Either:} The report may be freely copied and distributed provided the source is explicitly acknowledged
\newline  %% \\ screws it up
\emph{Or:}\newline
The report will be distributed to the internal and external examiners, but thereafter may not be copied or distrbuted except with permission from the author.}
\\ \\
Name of your degree\\ \\
Supervisor's name}



\begin{document}
 
 \onehalfspacing
\maketitle

% ABSTRACT 
\begin{abstract}
Summarise your report concisely.
\end{abstract}

% ACKNOWLEDGEMENTS
\renewcommand{\abstractname}{Acknowledgements}
\begin{abstract}
I would like to express my gratitude to my supervisor, Prof. Ralf Toumi for his valuable guidance and advice throughout the project. It was an eye-opening experience into how academic research is performed and with his patience and experience, it was easier to navigate the unfamiliar environment. 

Furthermore, I would like to thank the rest of the SPAT Group, Dr. Shuai Wang, Dr. Nathan Sparks and Thomas P Leahy for the discussions we had. In particular I would like to further extend gratitude to Dr. Shuai Wang for providing the base code for the extraction of data, which was of great help in the beginnings of the project. 

Finally, I would like to thank my project partner for her support, help, and thoughtful discussions throughout the project.
\end{abstract}


\tableofcontents
\setcounter{page}{1}

\listoffigures
\listoftables

%% Remember to use latex packages
\chapter{Glossary of Terms}

\chapter{List of Acronyms} 

%% CHAPTER 1
\chapter{Introduction}
• What problem are you solving?
• Why are you solving it?
• How does this relate to other work in this area?
• What work does it build on?
• What is the scope of your work?
• What is included in the scope and what is outside the scope?

- Unified process & gantt chart 

\section{Problem Statement}
% Right at the start, very clearly outline the problem you are working
% on, why it is interesting and what the challenges are. Make sure your
% reader is clear on what your project is about after reading the first
% few paragraphs. Don’t begin waffling on about how technology is
% changing the world, how great the internet is, or how you intend to
% revolutionise computing!



\section{Aims and Goals}
% List your aims and goals. An aim is something you intend to achieve
% (e.g., learn a new programming language and apply it in solving the
% problem), while a goal is something specific you expect to deliver
% (e.g., a working application with a particular set of features).

\section{Project Overview}
%{ Give an overview of how you carried out the project (e.g., an iterative
% approach). I think here you discuss your software development methodology. 

\section{Report Overview}

%{ A brief overview of the rest of the chapters in the report (a guide to
% the reader of the overall structure of the report).





%% CHAPTER 2
\chapter{Background \& Literature Review}
• What design choices did you have along the way, and why did you
make the choices you made?
• What was the most difficult part of the project?
• Why was it difficult? How did you overcome the difficulties?
• Did you discover anything novel?
• What did you learn?

%% CHAPTER 3 REQUIREMENTS AND ANALYSIS
\chapter{Requirements Gathering and Analysis}

- Use of MoSCow prioritisation 
- divide into functional/non-functional requirements table
- use case modelling  - usecase description and actors 
- use case diagram 
- entity relationship diagram for use cases
- data flow analysis + data ERD 

- personas / secnarios
- use case diagram 
- use case specifications 
- mock up  basamiq // we can just use figma 


\section{Requirements Gathering}
\subsection{Personas}
\subsection{Scenarios} 
\section{Requirements Analysis}
\subsection{Functional Requirement}
\subsection{Non-Functional Requirements}
\section{Use Case Diagram}
\section{Use Case Speciications} 
\section{Mock-Up}


%% CHAPTER 4 DESIGN AND IMPLEMENTATION
\chapter{Design and Implementation}

- system archiecture diagram Multiple layers - User Layer / application Layer / CLoud Layer / Data processing Layer / Data Layer 
- activity diagram 
- design patterns, database scehema (datagrip?)
- wireframe

olivia phillips 
- stepwise refinement 
- front end, back end , database  - discuss how each layer talks to each other 
- process flow diagram 


\section{Process Flow Diagram}
\section{High-Level Architecture}

%% TESTING
\chapter{Testing}

- test driven development 
- extreme pgraomming 

\section{Unit Testing}
\section{Integration Testing}
\section{System Testing}
\section{Acceptance Testing}

%% CONCLUSION
\chapter{Conclusion and Project Evaluation}

\section{Overview}

\section{Achievements}
\subsection{Project Achievements}
\subsection{Personal Achievements}

\section{Critical Evaluation}
\subsection{Functionality}
\subsection{Reliability}
\subsection{Usability}

\section{Future Steps}
\section{Final Thoughts}

\appendix

%% REFERENCES
\chapter{Bibliography}

\begin{thebibliography}{HHM99}


\bibitem[Pri70]{PriorNOP70}  %%only an example
A.~Prior.
\newblock The notion of the present.
\newblock {\em Studium Generale}, 23:  245--248, 1970.


\bibitem[Rey97]{Rey:D}
M.~Reynolds.
\newblock A decidable temporal logic of parallelism.
\newblock {\em Notre Dame Journal of Formal Logic}, 38(3):  419--436,
  1997.
\end{thebibliography}

%% APPENDIX
\chapter{Appendix}
\subsection{System Manual}
\subsection{User Manual}
\subsection{Supporting Documents and Diagrams}
\subsection{Test results and test reports}
\subsection{Evaluation Data and Results}
\subsection{Code Listing}

\end{document}